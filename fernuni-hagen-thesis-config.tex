% ****************************************************************************************************
% ferniunithesis-config.tex 
% Use it at the beginning of your thesis.tex, or as a LaTeX Preamble 
% in your thesis.{tex,lyx} with \input{fernunithesis-config}
% ****************************************************************************************************

% ****************************************************************************************************
% 1. Personal data and user ad-hoc commands
% ****************************************************************************************************
\newcommand{\myTitle}{Different query languages for graph databases\xspace}
\newcommand{\mySubtitle}{Research paper\xspace}
\newcommand{\myName}{Daniel Langhann\xspace}
\newcommand{\myId}{3788687\xspace}
\newcommand{\myProf}{Prof. Dr. Uta Störl\xspace}
\newcommand{\referent}{Mentoring\xspace}
\newcommand{\myFaculty}{Faculty of Mathematics and Computer Science\xspace}
\newcommand{\myUni}{FernUniversität Hagen\xspace}
\newcommand{\mySubjectArea}{Chair of Databases and Information Systems\xspace}
\newcommand{\myLocation}{Hagen\xspace}
\newcommand{\myTime}{29.04.2024\xspace}
\newcommand{\myVersion}{version 4.4\xspace}

% ****************************************************************************************************
% 2. Loading some handy packages
% ****************************************************************************************************
% ****************************************************************************************************
% Packages with options that might require adjustments
% ****************************************************************************************************

%\PassOptionsToPackage{ngerman,american}{babel}   % change this to your language(s)
% Spanish languages need extra options in order to work with this template
%\PassOptionsToPackage{spanish,es-lcroman}{babel}
\usepackage{babel}

