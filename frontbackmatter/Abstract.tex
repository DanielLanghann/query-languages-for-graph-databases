%*******************************************************
% Abstract in English
%*******************************************************
\pdfbookmark[0]{Abstract}{Abstract}


\begin{otherlanguage}{american}
	\chapter*{Abstract}
	Graph databases have become increasingly popular in recent years in the field of big data processing and in the area of 
	evaluations for large amounts of data.\\
	At the same time, the amount of available data is continuously increasing, and with it the need for technologies that are capable of
	efficiently store and query this ever-increasing amount of data and make it searchable.\\
	Graph databases are suitable for precisely this purpose.
	Various database technologies and query languages have been developed in the past for graph data and graph databases. At the same time, 
	an ISO standard for querying graph databases based on the \ac{SQL} standard was developed, 
	which was published in the process of this work under ISO/IEC 39075.\\ 
	The motivation of this work is to analyze selected query languages 
	and their underlying technologies and to make them comparable on the basis of defined criteria.\\ 
	The paper begins with a general introduction to the topic of graph data and graph databases in which, 
	among other things, important terms are clarified.
	The main section in chapter \ref{ch:different_query_languages_for_graph_databases} then presents three of the most popular query languages and 
	compares them with the previously introduced comparison criteria. 
	The paper concludes in chapter \ref{ch:conclusion} with
	a summary of the results of the analyses and gives a general outlook on how query languages 
	for and the use of graph databases could develop in the future.

\end{otherlanguage}
