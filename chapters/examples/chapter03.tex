\chapter{The GQL standard}
\label{ch:iso}
\section{Overview of the GQL standard}
\label{sec:iso:overview}
In short, GQL is a new standard for a property graph database language developed by 
the ISO committee.
It is the first time in more than 35 years, that the ISO released a new 
standard for database query language \citep{hare_isoiec_2024}.
In opposite to SQL, where data is organized in tables, graph databases structure data in graphs . 
This enables new ways to analyze and recognize patterns in very large amounts of 
data without having specific knowledge about the data itself \citep{PINO2024100997}.
Various use cases for graph databases have already been mentioned in chapter \ref{sec:intro:research_topic}, 
one other interesting example can be found in \citep{SAAD2023136344}.

\section{Scope of the GQL standard}
\label{sec:iso:scope}
Graph databases store and retrieve nodes (vertexes) and
edges between nodes (relationships). GQL is a delclarative language influenced both by 
existing property graph database products and by the SQL standard.
The GQL standard is a complete database language that supports, creating, reading
updating, deleting and modifying property graph data \citep{hare_isoiec_2024}. \\
Graph data can be organized in two different ways, Schema free or constrained by an database administrator.
Schema free graph data has no restrictions for adding and changing graph data.
Graph data that is subject to a predefined schema must fulfill this at all times, 
otherwise an error is triggered \citep{WIHARJA2020100616}.\\
The schema of graph data is specified via so-called graph types, 
which on the one hand specify the structure of the nodes as well as that of the edges, 
i.e. the relationships between the nodes \citep{hare_isoiec_2024}.

\section{Examples of common GQL query operations}
\label{sec:iso:examples}
This chapter provides a brief introduction to GQL capabilities including, 
Queries and \ac{GPM}, Add, modify and delete operations.

\subsection{Queries and Graph Pattern Matching}
\label{subsec:iso:examples:queries_and_graph_pattern_matching}
GQL queries are based on rich GPM language.
The below example findes all nodes with a one-hop relationship to a node with a 
with a specific \textit{productId} \citep{hare_isoiec_2024}:
\begin{lstlisting}[caption={Example for one hop graph pattern query}, label={lst:createOneHopGraph}]
    {
       MATCH (a {productId: "6594301c4aa9b3232889e7c3"}) - [b] -> (c)
	   RETURN a, b, c

    }
\end{lstlisting}
The GQL standard does not define how the returned data has to be displayed
to the user. Two possible ways can be, Graph Data Visualization tools and Textoutput.


\subsection{Quantified graph patterns}
\label{subsec:iso:examples:quantified_path_patterns}
The GPM of GQL also provides capabilities for \ac{QGP} \citep{fan_adding_2016}.
The following example finds all paths where one node is related to another node
up to ten hops:
\begin{lstlisting}[caption={Example for quantified graph pattern}, label={lst:createQuantifiedGraph}]
    MATCH((a)-[r]->(b)){1,10}
	RETURN a, r, b
\end{lstlisting}
More complex QGP are possible.

\subsection{Complex Graph pattern}
The listing \ref{lst:benchmark} matches pattern
with nodes of type \textit{Person} and \textit{Company}. 
The relationships between the nodes are \textit{worksFor} and \textit{partnerOf}. 
The matching pattern:
\begin{lstlisting}
    MATCH (person:Person)-[r1:worksFor]->(company:Company)-[r2:partnerOf]->(partner:Company)
\end{lstlisting}
matches a pattern where a \textit{Person} works for a \textit{Company} and that 
\textit{Company} has a partnership with another \textit{Company}.
The filter of the query:
\begin{lstlisting}
	WHERE person.age > 25 AND company.revenue > 100000
\end{lstlisting}
filters all persons with an age > 25 and companys with a revenue > 100000.
The part or WITH-Clause
\begin{lstlisting}
	WITH person, company, partner, COUNT(r1) AS work_relationships, COLLECT(r2) AS partneRelationships
\end{lstlisting}
aggregates the number of \textit{worksFor} relationships and collects the 
\textit{partnerOf} relationships for further use in the query in the second match pattern:
\begin{lstlisting}
	MATCH (person)-[r3:knows]->(colleague:Person)
	WHERE colleague.age < 30
\end{lstlisting}
\newpage

\subsection{Insert, Update and Delete operations}
\label{subsec:iso:examples:insert_update_and_delete_operations}
The following examplein listing \ref{lst:insertStatement} shows a typical insert statement in GQL:
\begin{lstlisting}[caption={Typical insert statement in GQL}, label={lst:insertStatement}]
    /*Insert one node */
	INSERT (:Customer {name: 'XYZ Ltd.', customerStatus: 'Active'})
\end{lstlisting}
In this example \textit{Customer} is a Label and \textit{name} and \textit{customerStatus}
are properties.\newline
Labels are identifiers that are either present or not present.
Properties are Key-Value pairs.\newline
Both nodes and relationships can have labels and properties \citep{PINO2024100997}.
Basically, nodes are enclosed in parenthesis:
\begin{lstlisting}
	(:Person {firstName: 'Daniel', lastName: 'Langhann'})
\end{lstlisting}
and relationships (edges) are enclosed in square brackets:
\begin{lstlisting}
	 [:r3:knows]
\end{lstlisting}
GQL supports insert operations for complex graph pattern like listing \ref{lst:complexInsertStatement}.
The statement in listing \ref{lst:complexInsertStatement} inserts two nodes \textit{Customer} and
\textit{City} and the relationship \textit{located}.\newline
Insert operations as shown in Listing \ref{lst:edgeInsertStatement} can also be the result of a \textit{MATCH} pattern:
\begin{lstlisting}[caption={Pattern based insert statement}, label={lst:edgeInsertStatement}] 
	/*Creates an edge isStudentOf. */
	MATCH (a {name: 'Langhann'}), (b {name: 'Stoerl'})
	INSERT (a)-[:isStudentOf]->(b)
\end{lstlisting}
In the Listing \ref{lst:edgeInsertStatement} the variables \textit{a} and \textit{b} are
aliases.\newline They are defined in the \textit{MATCH} clause and are only part of the Memory until
the Query determines.\newline 
The result of the  \textit{MATCH} caluse in listing \ref{lst:edgeInsertStatement} is a \textbf{cartesian product} of the two nodes.
This is why each node expression returns only one node. The \textit{INSERT} statement will only insert one edge.\newline
An Update in GQL is done by identififying the nodes or edges to be updated.
After identififying the instances the user can set or remove properties (listing \ref{lst:updateStatement}):
\begin{lstlisting}[caption={Update statements in GQL}, label={lst:updateStatement}] 
	/*Update*/
	MATCH (d:Customer) where d.id = '6594301c4aa9b3232889e7c3'
	SET d.status='inactive'
\end{lstlisting}
The following listing \ref{lst:removeStatement} removes the properties of a node \textit{Customer}:
\begin{lstlisting}[caption={Delete propteries in GQL}, label={lst:removeStatement}] 
	/*Remove propertiers*/
	MATCH (d:Customer) where d.id = '6594301c4aa9b3232889e7c3'
	REMOVE d.invoiceAccepted
\end{lstlisting}
In listing \ref{lst:deleteStatement} a node and the related nodes where deleted:
\begin{lstlisting}[caption={Remove nodes in GQL cascading}, label={lst:deleteStatement}] 
	/*Delete Customer and related Nodes*/
	MATCH  (a {id: '6594301c4aa9b3232889e7c3'})-[b]->(c)
	DETACH DELETE a,c
\end{lstlisting}
\section{Transactions}
\label{sec:iso:transactions}
In GQL serializable transactions and their additional implmentation-defined transaction modes 
are supported.
Transactions are starting with either an \textbf{explicit} or \textbf{implicit} \textit{START TRANSACTION}
statement and terminated with either a \textit{COMMIT} or \textit{ROLLBACK}.
The user can also  implement automatic transactions starts and commits \citep{hare_isoiec_2024}.

\section{Schema free vs graph types}
\label{sec:iso:schema_free_vs_graph_types}
A schema free graph accepts any form of graph data which makes it relatively fast to use. 
But on the other hand it gives the user control over the data 
and therefore a certain amount of data proliferation is accepted and has to be managed \\
A graph type is a kind of template that specifies the structure of the graph, which must be adhered to at all times.
The structure of the nodes as well as the edges or both can be specified \citep{10565888}.
The listing \ref{lst:graphType} illustrates the concept of a graph type.
If the user is creating a graph by using a graph type, the contents of the nodes and edges 
are constrained by the graph type, which makes sure the data model will be respected in 
all transactions. This is also relevant for update operations. If an update operation does 
not follow the restrictions of the graph type, the transactin will be rolled back and 
the database system will throw an error \citep{10565888}.

\section{Compare graph structured data and tables}
\label{sec:iso:compare_graph_structured_data_and_tables}
In a typical SQL database, data is organized and stored in tables and rows where each 
table has a fixed schema.
Relationships between tables are basically defined as so called Foreign Keys \citep{1702016}.
The user of the database has to know and understand these relationships to be able
to write queries over more than one table.\newline
In a property graph database, the level of abstraction is high, and allows the user 
whole sets of tables to treat as one unit, which can be understand as a data product \citep{PINO2024100997}.
