\chapter{introduction}
\label{ch:intro}

\section{Research topic}
\label{sec:intro:Forschungsthema}
Graph databases have developed into a powerful tool for modeling and querying com-plex relations in various data types. 
They are particularly suitable for dealing with linked data and complex network structures that essentially consist of nodes and their edges (relations). 
Examples of use are e.g \citep{yuanyuan_tian_world_2022}:
\begin{itemize}
	\item Relationships in social networks
	\item Recommendation applications
	\item Biological networks
\end{itemize}
In this paper, the topic of graph databases and selected query languages designed for graph databases will be presented and contextualized. 
In particular, the path from different query languages to a uniform ISO standard is pointed out.


\section{Motivation}
\label{sec:intro:Motivation}
The aim of this paper is to provide an overview of the different languages and to highlight the respective advantages and disadvantages, 
their capabilities, and limitations, as well as their specific use cases. 

\section{State of research}
\label{sec:intro:State of researc}
Graph databases are a comparatively young technology in the field of database systems. 
Over time, various query languages have been developed with specific focuses and corresponding strengths and weaknesses.
A common ISO standard should help to keep the technology and its application more generic \citep{hare_isoiec_2024}.
In this paper, important proprietary languages are presented and differentiated from each other. 
In the further course, a overview of the different languages is created.

\section{Approach}
\label{sec:intro:Approach}
Basically, relevant documentation on the various query languages is analyzed and the key points are highlighted. 
To support this, selected languages are tried out in a test project and the corresponding learnings are integrated into the work.


\chapter{Criteria for comparison}
First, comparative criteria are introduced against which the different languages are compared.

The evaluation is done purely qualitatively as this is a complex topic that is difficult to evaluate on the basis of a uniform scale. 
Nevertheless, the aim is to evaluate the languages as objectively and reproducibly as possible on the basis of the defined criteria.

\section{Expressiveness}
Expressiveness in the context of query languages for graph databases means the ability 
of a language to allow a wide range of queries on graphs and their underlying databases.
An expressive language can be used with
\begin{itemize}
	\item complex patterns
	\item relationships
\end{itemize}
even across various different graphs.
In addition, an expressive language offers users the corresponding flexibility
when querying graph structures.

\section{Ease of Use}
For users, it is important that the corresponding query language is intuitive to understand, follows familiar and learned patterns 
and is as close as possible to the de facto industry standard. 
On the other hand, a less intuitive language means an increased learning curve, 
which makes the technology more difficult to use.

\section{Performance and Performance Optimization}
Performance is a decisive factor when dealing with the analysis of large amounts of data. 
Especially when it comes to evaluations that need to be carried out in real time or very quickly. 
A recommendation system does not help anyone if it does not deliver an answer in a reasonable time.
This criterion evaluates in particular the ability of the corresponding language to enable techniques 
that improve the performance of queries on graph databases.
Particular attention is paid to the following two factors:
* Implementation of index-based structures
* Parallel processing

\section{Interoperability}
How well a query language intergrates with other tools, techniques and API`s is an important
criteria for the decision for and against the query language and often for the entire 
database system.

\section*{Closeness to the ISO standard}
n particular, attention is paid to the extent to which the respective language is oriented 
towards the ISO standard for query languages for graph databases.
A language that is strongly oriented towards the ISO standard or whose model and structure 
has a high proportion of the ISO standard automatically ensures, 
that the user has to acquire a low level of proprietary knowledge, 
which is beneficial to the user-friendliness of the language.
In addition, a strong orientation towards the ISO standard ensures interoperability 
with other technologies and systems, 
as it can be assumed that these are also oriented towards at least the basic concepts of the ISO standard.

In the next chapter the ISO-standard for Graph databases will be introduced to enable the reader 
to compare the corresponding language with the ISO-standard when reading chapter XXX.

\chapter*{The ISO-standard for query languages for graph databases}




\chapter{Timeline}
\label{sec:intro:Timeline}


