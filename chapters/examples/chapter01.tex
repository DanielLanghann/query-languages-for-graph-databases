\chapter{Introduction}
\label{ch:intro}

\section{Research topic}
\label{sec:intro:research_topic}
Graph databases‚have developed into a powerful tool for modeling and querying complex relations in various data types. 
They are particularly suitable for dealing with linked data and complex network structures that essentially consist of nodes and their edges (relations). 
Examples of use are e.g, Relationships in business networks, Recommendation applications
and Biological networks \citep{yuanyuan_tian_world_2022}.
In this paper, the topic of graph databases and selected query languages designed for graph databases will be presented and contextualized. 
In particular, 
the path from different query languages to a uniform \ac{GQL} standard by the \ac{ISO} is pointed out \citep{hare_isoiec_2024}.
Further the term \ac{GQL} standard is used. 

\section{Motivation}
\label{sec:intro:motivation}
The aim of this paper is to provide an overview of selected important 
query languages for graph databases and to highlight the respective advantages and disadvantages, 
their capabilities, and limitations, as well as their specific use cases. 

\section{State of research}
\label{sec:intro:state_of_research}
Graph databases are a comparatively young technology in the field of database systems \citep{hare_isoiec_2024}.  
Over time, various query languages have been developed with specific focuses and corresponding strengths and weaknesses.
A common ISO standard should help to keep the technology and its application more generic \citep{duckham_matt_gis_2024}.
In this paper, important proprietary languages are presented and differentiated from each other. 
In the further course, an overview of the different languages is created.

\section{Approach}
\label{sec:intro:approach}
Basically, relevant documentation on the various query languages is analyzed 
and the key points are highlighted. 
To support this, selected languages are tried out in a test project 
and the corresponding learnings are integrated into the work.
