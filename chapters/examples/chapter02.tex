\chapter{Criteria for comparison}
\label{ch:criteria_for_comparison}
First, comparative criteria are introduced against which the different languages are compared.
The evaluation is done purely qualitatively as this is a complex topic that is difficult to evaluate on the basis of a uniform scale. 
Nevertheless, the aim is to evaluate the languages as objectively and reproducibly as possible on the basis of the defined criteria.

\section{Expressiveness}
\label{sec:criteria_for_comparison:expressiveness}
Expressiveness in the context of query languages for graph databases means the ability 
of a language to allow a wide range of queries on graphs and their underlying databases.
An expressive language can be used with complex patterns and relationships
even across various different graphs.
In addition, an expressive language offers users the corresponding flexibility
when querying graph structures \citep{barcelo_expressive_2012}.

\section{Performance and Performance Optimization}
\label{sec:criteria_for_comparison:performance}
Performance is a decisive factor when dealing with the analysis of large amounts of data. 
Especially when it comes to evaluations that need to be carried 
out in real time or very quickly \citep{yuanyuan_tian_world_2022}. 
A recommendation system does not help anyone if it does not deliver 
an answer in a reasonable time \citep{MONDAL2020103549}.
This criterion evaluates in particular the ability 
of the corresponding language to enable techniques 
that improve the performance of queries on graph databases \citep{yuanyuan_tian_world_2022}.


\section{Interoperability}
\label{sec:criteria_for_comparison:interoperability}
How well a query language intergrates with other tools, techniques and other API`s (\ac{API}) is an important
criteria for the decision for and against the query language and often for the entire 
database system \citep{timon-reina_overview_2021}.

\section{Closeness to the GQL standard}
\label{sec:criteria_for_comparison:iso}
Attention is also paid to how closely each language aligns with the 
GQL standard. 
A language that closely follows this standard, 
or whose model and structure incorporate a significant portion of it, 
ensures that users need to acquire minimal proprietary knowledge, 
thereby enhancing user-friendliness. 
Moreover, strong alignment with the GQL standard promotes interoperability 
with other technologies and systems, as these are likely based on at least 
the fundamental concepts of GQL \citep{hare_isoiec_2024}.
The next chapter introduces the GQL standard, allowing the reader to compare various languages with it when reading chapter 
\ref{ch:different_query_languages_for_graph_databases}.

\section{Benchmark Querie}
The benachmark querie to compare the capabilities of the different languages can be found in 
listing \ref{lst:benchmark}.

